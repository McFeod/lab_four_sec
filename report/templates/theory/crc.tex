Циклические коды основаны на полиноминальной арифметике по модулю 2 (полиноминальном делении без переноса).
Вместо представления делимого (исходного сообщения, входных данных), делителя (порождающего полинома), частного
(целой части) и остатка (контрольной суммы, CRC) в виде положительных целых чисел, их можно представить в виде полиномов
 с двоичными коэффициентами или в виде строки битов, каждый из которых является коэффициентом полинома.
Например, десятичное число $19_{10}$ в двоичной системе счисления имеет вид $10011_2$, что совпадает с полиномом

$ 1 \cdot x^4 + 0 \cdot x^3 + 0 \cdot x^2 + 1\cdot x^1 + 1 \cdot x^0 = x^4 + x^1 + x0$.

Значение контрольной суммы с порождающим полиномом $G(x)$ определяется по формуле:

$R(x) = P(x) \cdot  x^N mod G(x)$, 

где $R(x)$ \-- полином, представляющий значение контрольной суммы;
$P(x)$ \-- полином, представляющий входные данные;
$G(x)$ \-- порождающий полином;
$N$ \-- максимальная степень порождающего полинома.
 Умножение $x^N$ эквивалентно приписыванию N нулевых битов к входным данным.
Полиноминальное деление без переноса выполняется по следующим правилам:
\begin{itemize}
\item при наличии у промежуточного остатка в качестве старшего бита «1», он складывается по модулю 2 (XOR, исключающее ИЛИ) с битовым представление порождающего полинома и в частное записывается «1»;
\item в противном случае выполняется сложение по модулю 2 промежуточного остатка с нулевой битовой строкой длиной N+1 и в частное записывается «0».
\end{itemize}