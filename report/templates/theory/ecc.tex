ECC являются разновидностью кодов Хэмминга, обозначаемых обычно $(m, x)$, где $m$ \--
 количество информационных и контрольных битов в блоке данных, $x$ \-- количество информационных битов.
Формула классического кода Хэмминга имеет вид $(2r - 1, 2r – r – 1)$, где $r$ \-- количество контрольных
битов (например, (7, 4), (15, 11) или (31, 26)). Данная формула позволяет обнаруживать одиночную или
двойную ошибку, а также исправлять одиночную ошибку, но при условии, что ошибок не может быть более двух.

Вычисление последовательности контрольных битов:
\begin{enumerate}
\item Контрольные биты устанавливаются в 0;
\item Контрольные биты вставляются в исходное сообщение на позиции с номерами $2^i$, $i = 0, 1, 2,..$,
получается последовательность $XR$;
\item Составляется матрица $N$:
\begin{itemize}
    \item Число строк соответствует количеству контрольных битов;
    \item В столбцы добавленных строк записываются двоичные представления номеров позиций битов,
    причем порядок следования битов двоичного представления будет обратный \-- младший бит
    располагается в верхней строке, старший \-- в нижней;
\end{itemize}
\item Вычисляются коннтрольные биты по формуле $r_j = (\sum_{i=1}^{m} XR_i \cdot N_{ij}) \ mod \ 2$, где:
\begin{itemize}
    \item $j$ \-- номер строки;
    \item $i$ \-- номер столбца;
    \item $m$ \-- число столбцов;
\end{itemize}

\end{enumerate}

Проверка целостности
\begin{enumerate}
\item Контрольные биты вставляются в исходное сообщение на позиции с номерами $2^i$, $i = 0, 1, 2,..$,
получается последовательность $XR$;
\item Составляется матрица $N$;
\item Вычисляется нечетный паритетный бит $pb$;
\item Вычисляется вектор синдромов $S = (XR \cdot N^T) mod 2$;
\item Если $pb = 0$ и $S$ состоит из нулей, считается, что ошибки нет.
\item Если $pb = 0$ и $S$ не состоит из нулей, значит при передече возникло 2 ошибки.
\item Если $pb \neq 0$ и $S$ не состоит из нулей, то считается, что ошибка одна. Переставив биты из вектора синдромов
$S$ в обратном порядке получают номер ошибочного бита, который можно исправить, инвертировав его.
\end{enumerate}