\begin{enumerate}
\item Вычисляется сумма цифр, стоящих на нечётных позициях $S_e = \sum{d_{2_i + 1}}$ (кроме последней, вычисляемой);
\item Вычисляется утроенная сумма цифр, стоящих на чётных позициях $S_o = 3 \cdot \sum{d_{2_i}}$ (кроме последней, вычисляемой);
\item Вычисляется последяя цифра $cd$, такая, что $(S_e + S_o + cd) \ mod \ 10 = 0$.

\end{enumerate}

Если количество цифр в коде нечетное, то 1 и 2 операция выполняются для цифр,
 стоящих в четных позициях, 3 операция \-- для цифр, стоящих в нечетных позициях.